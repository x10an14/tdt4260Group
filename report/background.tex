\section{Background}

\subsection*{DCPT}
\todo[inline]{Beskrivelse av DCPT.}

The size of the prefetcher has been a limitation in our development. Hardware
components must usually be designed within an area budget, and in order to make
the prefetcher somewhat realistic for a hardware implementation the limit was
set to maximum 8 KB of storage \cite{guidelines}. (Må redgjøre for at vi ikke overskrider denne et
sted i rapporten?) The DCPT utilizes a table of ring buffers for keeping track
of the deltas, and so organizing the memory effectively is an important aspect.
In the original implementation, only 4 KB of storage is used \cite{dcpt}.

\subsection*{Tagged Sequential Prefetching}

The tagged sequential prefetcher a simple prefetching method, slightly improving
the performance of a sequential prefetcher \cite{grannaes}. As with sequential
prefetching the spatial locality of data is exploited.

\todo[inline]{Hvor mye av beskrivelsen skal her? Og hvor mye i Prefetcher Description?}

\todo[inline]{Related Work\\
- Reference the work that other researchers have done
that is related to your scheme\\
- Should be complete (i.e. contain all relevant work)\\
- Remember: you define the scope of your work\\
Can be split into two sections: Background and Related
Work\\
- Background is an informative introduction to the field (often section 2)\\
- Related work is a very dense section that includes all relevant
references (often section n-1)\\}
