\section{Background}

In this section, we go over the algorithmic details of the two prefetchers we
have chosen to implement, detailing how our implemented prefetchers are intended
to work.

\subsection{TS Prefetching Algorithm}

The simplest prefetcher is the Sequential prefetcher, which fetches the next
block of data \cite{seq}. The TS prefetcher is a slight
improvement over the Sequential prefetcher. As with Sequential prefetching the
spatial locality of data is exploited.

Additionally, a tagging system is used to mark the cache blocks that have been
prefetched. This system is used to recognize when a prefetched cache block is
accessed by the application, which implies that there has been a successful
prefetch. Subsequent blocks are prefetched into the cache when there has been a
cache miss, or when a previously prefetched cache entry has been
accessed~\cite{grannaes}.

The TS prefetcher can be configured by adjusting two parameters;
the \emph{distance} is how many blocks ahead of the currently accessed block the
prefetcher will fetch, and the \emph{degree} is how many consequtive blocks it
will fetch starting at this location in memory.

\subsection{Delta Correlating Prefetching Table Algorithm}

The DCPT prefetcher uses a more advanced algorithm. It is an instruction-based
prefetcher using a table indexed by the address of the instruction which tried
to access memory.

The DCPT utilizes a table of ring buffers for keeping track of the deltas, and
so organizing the memory effectively is an important aspect. In the original
implementation~\cite{dcpt} we based our implementation on, only 4KiB of storage
are used.

Each entry contains a ring buffer of the most recent deltas between successive
addresses accessed by this instruction.

When an access happens, the prefetcher
examines the two most recent deltas and searches the ring buffer for patterns
beginning with these two deltas. If there are multiple matches, we choose the
longest matching sequence of deltas.

The prefetcher then adds the found delta pattern to the current address to
obtain the predicted stream of future accesses. It discards addresses which
are already in the cache, or ones that are already being prefetched. It then
issues prefetch commands for the remaining addresses.

\todo[inline]{Related Work\\
- Reference the work that other researchers have done
that is related to your scheme\\
- Should be complete (i.e. contain all relevant work)\\
- Remember: you define the scope of your work\\
Can be split into two sections: Background and Related
Work\\
- Background is an informative introduction to the field (often section 2)\\
- Related work is a very dense section that includes all relevant
references (often section n-1)\\}
