\section{Background}

\subsection*{Tagged Sequential Prefetching}

The tagged sequential prefetcher is a slight improvement over the sequential
prefetcher. As with sequential prefetching the spatial locality 
of data is exploited. Additionally, a tagging system is used to mark the cache 
blocks that have been prefetced. This is used to be able to recognize when a 
prefetched cache block is actually accessed by the application, which implies 
that there has been a successful prefetch. Subsequent blocks are prefetched 
to the cache when there has been a cache miss, or when a previously prefetched 
cache entry has been accessed.\cite{grannaes}

\subsection*{DCPT}
\todo[inline]{Beskrivelse av DCPT.}

DCPT is a more advanced method. It is an instruction-based prefetcher using a
table indexed by the address of the instruction which tried to access memory.
Each entry contains a ring buffer of the most recent deltas between successive
addresses accessed by this instruction. When an access happens, the prefetcher
examines the two most recent deltas and searches the ring buffer for patterns
beginning with these two deltas. If there are multiple matches, we choose the
longest match.

The prefetcher then adds the found delta pattern to the current address to obtain
the predicted stream of future accesses. It then discards addresses which are
already in the cache, or ones that are already being prefetched. It then issues
prefetch commands for the remaining addresses.

\todo[inline]{Størrelse og andre parametre bør heller være i metodologi-delen.}

The size of the prefetcher has been a limitation in our development. Hardware
components must usually be designed within an area budget, and in order to make
the prefetcher somewhat realistic for a hardware implementation the limit was
set to maximum 8 KB of storage \cite{guidelines}. (Må redgjøre for at vi ikke overskrider denne et
sted i rapporten?) The DCPT utilizes a table of ring buffers for keeping track
of the deltas, and so organizing the memory effectively is an important aspect.
In the original implementation, only 4 KB of storage is used \cite{dcpt}.


\todo[inline]{Hvor mye av beskrivelsen skal her? Og hvor mye i Prefetcher Description?}

\todo[inline]{Related Work\\
- Reference the work that other researchers have done
that is related to your scheme\\
- Should be complete (i.e. contain all relevant work)\\
- Remember: you define the scope of your work\\
Can be split into two sections: Background and Related
Work\\
- Background is an informative introduction to the field (often section 2)\\
- Related work is a very dense section that includes all relevant
references (often section n-1)\\}
