\section{Background}

\todo[inline]{We need a small ingress here \\
-Nico}

\subsection{Tagged Sequential Prefetching Algorithm}

The simplest prefetcher is the Sequential prefetcher, which simply fetches the
next block of data \cite{seq}. The Tagged Sequential prefetcher is a slight
improvement over the Sequential prefetcher. As with sequential prefetching the
spatial locality of data is exploited. Additionally, a tagging system is used to
mark the cache blocks that have been prefetced. This is used to be able to
recognize when a pre-fetched cache block is actually accessed by the
application, which implies that there has been a successful prefetch. Subsequent
blocks are prefetched into the cache when there has been a cache miss, or when a
previously prefetched cache entry has been accessed~\cite{grannaes}.

\subsection{Delta Correlating Prefetching Table Algorithm}

DCPT is a more advanced method. It is an instruction-based prefetcher using a
table indexed by the address of the instruction which tried to access memory.
Each entry contains a ring buffer of the most recent deltas between successive
addresses accessed by this instruction. When an access happens, the prefetcher
examines the two most recent deltas and searches the ring buffer for patterns
beginning with these two deltas. If there are multiple matches, we choose the
longest match.

The prefetcher then adds the found delta pattern to the current address to
obtain the predicted stream of future accesses. It discards addresses which
are already in the cache, or ones that are already being prefetched. It then
issues prefetch commands for the remaining addresses.

\todo[inline]{What is prefetch degree and distance!? \\
-Nico}

\todo[inline]{Related Work\\
- Reference the work that other researchers have done
that is related to your scheme\\
- Should be complete (i.e. contain all relevant work)\\
- Remember: you define the scope of your work\\
Can be split into two sections: Background and Related
Work\\
- Background is an informative introduction to the field (often section 2)\\
- Related work is a very dense section that includes all relevant
references (often section n-1)\\}
