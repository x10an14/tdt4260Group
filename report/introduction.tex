\section{Introduction}

\IEEEPARstart{T}{he} memory wall is becoming an increasing challenge for the
computers of tomorrow. The problem occurs when the on-chip speed of computations
exceed the on-chip memory capacities, creating the situation where computing
circuitry has to lie idle waiting for instructions/data to arrive from off-chip
memory locations. The cache is generally considered the quickest, most
expensive, and smallest memory, usually residing as closely on-chip as possible.
Because of this, a time-gap occurs when the on-chip circuitry needs to wait some
clock-cycles for the the correct instructions/data to be fetched from off-chip
memory, and into the cache of the chip requiring the instructions/data for
computations. \emph{Prefetching} is a method for reducing this gap.

The idea behind prefetching is to have a predictive module moving
instructions/data from the slower off-chip memory to the much faster cache
memory on-chip \emph{before} it is needed. Hence, when the processor requires
new instructions/data, they are already in the cache and can be made use of
within a small amount of on-chip clock cycles, isntead of having to wait for the
slower off-chip.

The purpose of this report is to analyze and compare a DCPT prefetcher with a
Tagged Sequential prefetcher using the M5 simulator, where both prefetchers
utilize a maximum of 8KiB. The DCPT prefetcher is implemented according to the
algorithm presented in ``Storage Efficient Hardware Prefetching using Delta
Correlation Prediction Tables'' by Grannaes, Jahre, and Natvig~\cite{dcpt}. The
storage limitation makes it important to examine how the 8 KiB can be used most
efficiently.

\todo[inline]{We need a small summary of the scheme chapter in the introduction as well!}

\todo[inline]{About the introduction:
- Introduces the larger research area that the paper is
a part of\\
- Introduces the problem at hand\\
- Explains the scheme\\}
