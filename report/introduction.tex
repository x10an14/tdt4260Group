\section{Introduction}

\IEEEPARstart{A}{n} increasing challenge in creating the computers of tomorrow
is called the ``memory wall''. This states that processor performance increases
much faster than the delay for accessing memory is decreasing. One method for
reducing this gap is called prefetching. The idea is to have a predictive module
moving  instructions/data from the slow main memory to the much faster cache
\emph{before} it is actually needed. When the processor then demands the
instruction/data, it is already in the cache and can be used right away.


The purpose of this project has been to develop and evaluate a prefetcher using
the M5 simulator, utilizing maximum 8KB of prefetching memory. We decided to
implement a prefetcher with Delta Correlating Prediction Table (DCPT), as this
design has proved to be very effective (reference), and it seemed reasonable
that we would be able to implement it with the given constraint. The
implementation is done according to the algorithm presented in "Storage
Efficient Hardware Prefetching using Delta Correlation Prediction Tables” by
Marius Grannaes, Magnus Jahre and Lasse Natvig. The storage limitation makes it
important to examine how the 8 KB can be used most efficiently. Possible
physical implementation will be discussed(?).


\hfill \today

\todo[inline]{About the introduction:
- Introduces the larger research area that the paper is
a part of\\
- Introduces the problem at hand\\
- Explains the scheme\\}