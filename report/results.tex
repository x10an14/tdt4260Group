\section{Results}

\subsection{Tagged sequential}
As described previously, the tagged sequential prefetcher was 
implemented with degree 5 and distance 4, as the performance 
with this configuration proved to be relatively good. The 
resulting average speedup was 1.032. The "ammp" and "twolf" benchmarks gave 
really bad results, with a speedup of respectively 0.759 and 0.983 
(speedup below zero means the prefetcher actually slows down the application). This is
expected as those applications use large objects and irregular cache alignment, 
which makes it very challenging to predict future memory accesses using spatial locality.

\todo[inline]{The results goes here.
\\
- Show that your scheme works\\
- Compare to other schemes that do the same thing. Hopefully you are better, but you need to compare anyway\\
Trick: “Oracle Scheme”\\
- Uses “perfect” information to create an upper bound on the
performance of a class of schemes\\
- Prefetching: Best case is that all L2 accesses are hits\\
Sensitivity analysis: \\
- Check the impact of model assumptions on your
scheme}