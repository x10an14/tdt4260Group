\section{Results}

\subsection{Tagged Sequential Results}

The Tagges Sequential prefetcher was simulated with different combinations of
degree and distance, in the range of 0--6. The combination that got the best
overall score on the benchmark tests was with degree 5 and distance 4.

The resulting average speedup was 1.032. The ``ammp'' and ``twolf'' benchmarks
gave lowest speedups, with a speedup of 0.759 and 0.983 respectively (speedup
results below 1.000 means the prefetcher actually slows down the application).

This is expected as those applications need to fetch instructions/data from random locations, making it very difficult for the Tagged Sequential prefetcher to get any use from sequential fetching, which relies on spatial locality.

\subsection{DCPT Results}

\todo[inline]{The results goes here.
\\
- Show that your scheme works\\
- Compare to other schemes that do the same thing. Hopefully you are better, but you need to compare anyway\\
Trick: “Oracle Scheme”\\
- Uses “perfect” information to create an upper bound on the
performance of a class of schemes\\
- Prefetching: Best case is that all L2 accesses are hits\\
Sensitivity analysis: \\
- Check the impact of model assumptions on your
scheme}
