\section{Results}
\label{sec:res}

\subsection{TS Results}
\todo[inline]{Graf her}

As can be observed in the graph, the TS configuration with the best
overall benchmark score has degree 1 and distance 9. The speedup with this configuration is 1.031. In general, degree 1 gives decent results, except with distance 10. Increasing the degree only gives worse performance.

The ``\emph{ammp}'' and ``\emph{twolf}'' benchmarks
gave the lowest speedups, with a speedup of 0.759 and 0.983 respectively (speedup
results below 1.000 means the prefetcher actually slows down the application).

This is expected as those applications need to fetch instructions/data from
random locations, making it very difficult for the TS prefetcher
to get any benefit from sequential fetching, which relies on spatial locality.

\subsection{DCPT Results}

\begin{table}[h]
\begin{tabular}{lllllll}
32    & 33    & 34    & \textbf{35}    & 36    & 37    & 38    \\
1.028 & 1.029 & 1.029 & \textbf{1.033} & 1.027 & 1.028 & 1.027
\end{tabular}
\end{table}

\todo[inline]{The results goes here.
\\
- Show that your scheme works\\
- Compare to other schemes that do the same thing. Hopefully you are better, but you need to compare anyway\\
Trick: “Oracle Scheme”\\
- Uses “perfect” information to create an upper bound on the
performance of a class of schemes\\
- Prefetching: Best case is that all L2 accesses are hits\\
Sensitivity analysis: \\
- Check the impact of model assumptions on your
scheme}
