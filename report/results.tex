\section{Results}

\subsection{Tagged sequential}

As described previously, the tagged sequential prefetcher was  implemented with
degree 5 and distance 4. The resulting average speedup was 1.032. The ``ammp''
and ``twolf'' benchmarks gave low speedups, with a speedup of
0.759 and 0.983 respectively (speedup below one means the prefetcher actually
slows down the application).

\todo[inline]{In the below sentence, what is meant by ``large objects''? We also
cannot predict future memory accesses using spatial locality. Re-phrase or
remove.}

This is expected as those applications use large
objects and irregular cache alignment,  which makes it very challenging to
predict future memory accesses using spatial locality.

\todo[inline]{The results goes here.
\\
- Show that your scheme works\\
- Compare to other schemes that do the same thing. Hopefully you are better, but you need to compare anyway\\
Trick: “Oracle Scheme”\\
- Uses “perfect” information to create an upper bound on the
performance of a class of schemes\\
- Prefetching: Best case is that all L2 accesses are hits\\
Sensitivity analysis: \\
- Check the impact of model assumptions on your
scheme}

Our prefetcher was simulated with different combinations of degree and distance,
in the range of 0--6. The combination that got the best overall score on the
benchmark tests was with degree 5 and distance 4.

\begin{comment}
\begin{figure}[tbp]
\begin{center}
    \input{tagged-sequential-plot}
    \caption{Speedup using the tagged sequential prefetcher}
    \label{graph:tagged-sequential}
\end{center}
\end{figure}

\begin{figure}[tbp]
\begin{center}
    % GNUPLOT: LaTeX picture with Postscript
\begingroup
  \makeatletter
  \providecommand\color[2][]{%
    \GenericError{(gnuplot) \space\space\space\@spaces}{%
      Package color not loaded in conjunction with
      terminal option `colourtext'%
    }{See the gnuplot documentation for explanation.%
    }{Either use 'blacktext' in gnuplot or load the package
      color.sty in LaTeX.}%
    \renewcommand\color[2][]{}%
  }%
  \providecommand\includegraphics[2][]{%
    \GenericError{(gnuplot) \space\space\space\@spaces}{%
      Package graphicx or graphics not loaded%
    }{See the gnuplot documentation for explanation.%
    }{The gnuplot epslatex terminal needs graphicx.sty or graphics.sty.}%
    \renewcommand\includegraphics[2][]{}%
  }%
  \providecommand\rotatebox[2]{#2}%
  \@ifundefined{ifGPcolor}{%
    \newif\ifGPcolor
    \GPcolorfalse
  }{}%
  \@ifundefined{ifGPblacktext}{%
    \newif\ifGPblacktext
    \GPblacktexttrue
  }{}%
  % define a \g@addto@macro without @ in the name:
  \let\gplgaddtomacro\g@addto@macro
  % define empty templates for all commands taking text:
  \gdef\gplbacktext{}%
  \gdef\gplfronttext{}%
  \makeatother
  \ifGPblacktext
    % no textcolor at all
    \def\colorrgb#1{}%
    \def\colorgray#1{}%
  \else
    % gray or color?
    \ifGPcolor
      \def\colorrgb#1{\color[rgb]{#1}}%
      \def\colorgray#1{\color[gray]{#1}}%
      \expandafter\def\csname LTw\endcsname{\color{white}}%
      \expandafter\def\csname LTb\endcsname{\color{black}}%
      \expandafter\def\csname LTa\endcsname{\color{black}}%
      \expandafter\def\csname LT0\endcsname{\color[rgb]{1,0,0}}%
      \expandafter\def\csname LT1\endcsname{\color[rgb]{0,1,0}}%
      \expandafter\def\csname LT2\endcsname{\color[rgb]{0,0,1}}%
      \expandafter\def\csname LT3\endcsname{\color[rgb]{1,0,1}}%
      \expandafter\def\csname LT4\endcsname{\color[rgb]{0,1,1}}%
      \expandafter\def\csname LT5\endcsname{\color[rgb]{1,1,0}}%
      \expandafter\def\csname LT6\endcsname{\color[rgb]{0,0,0}}%
      \expandafter\def\csname LT7\endcsname{\color[rgb]{1,0.3,0}}%
      \expandafter\def\csname LT8\endcsname{\color[rgb]{0.5,0.5,0.5}}%
    \else
      % gray
      \def\colorrgb#1{\color{black}}%
      \def\colorgray#1{\color[gray]{#1}}%
      \expandafter\def\csname LTw\endcsname{\color{white}}%
      \expandafter\def\csname LTb\endcsname{\color{black}}%
      \expandafter\def\csname LTa\endcsname{\color{black}}%
      \expandafter\def\csname LT0\endcsname{\color{black}}%
      \expandafter\def\csname LT1\endcsname{\color{black}}%
      \expandafter\def\csname LT2\endcsname{\color{black}}%
      \expandafter\def\csname LT3\endcsname{\color{black}}%
      \expandafter\def\csname LT4\endcsname{\color{black}}%
      \expandafter\def\csname LT5\endcsname{\color{black}}%
      \expandafter\def\csname LT6\endcsname{\color{black}}%
      \expandafter\def\csname LT7\endcsname{\color{black}}%
      \expandafter\def\csname LT8\endcsname{\color{black}}%
    \fi
  \fi
  \setlength{\unitlength}{0.0500bp}%
  \begin{picture}(5040.00,3528.00)%
    \gplgaddtomacro\gplbacktext{%
      \csname LTb\endcsname%
      \put(1210,704){\makebox(0,0)[r]{\strut{} 0.995}}%
      \put(1210,1070){\makebox(0,0)[r]{\strut{} 1}}%
      \put(1210,1435){\makebox(0,0)[r]{\strut{} 1.005}}%
      \put(1210,1801){\makebox(0,0)[r]{\strut{} 1.01}}%
      \put(1210,2166){\makebox(0,0)[r]{\strut{} 1.015}}%
      \put(1210,2532){\makebox(0,0)[r]{\strut{} 1.02}}%
      \put(1210,2897){\makebox(0,0)[r]{\strut{} 1.025}}%
      \put(1210,3263){\makebox(0,0)[r]{\strut{} 1.03}}%
      \put(1342,484){\makebox(0,0){\strut{} 0}}%
      \put(1672,484){\makebox(0,0){\strut{} 5}}%
      \put(2002,484){\makebox(0,0){\strut{} 10}}%
      \put(2332,484){\makebox(0,0){\strut{} 15}}%
      \put(2662,484){\makebox(0,0){\strut{} 20}}%
      \put(2993,484){\makebox(0,0){\strut{} 25}}%
      \put(3323,484){\makebox(0,0){\strut{} 30}}%
      \put(3653,484){\makebox(0,0){\strut{} 35}}%
      \put(3983,484){\makebox(0,0){\strut{} 40}}%
      \put(4313,484){\makebox(0,0){\strut{} 45}}%
      \put(4643,484){\makebox(0,0){\strut{} 50}}%
      \put(176,1983){\rotatebox{-270}{\makebox(0,0){\strut{}Speedup}}}%
      \put(2992,154){\makebox(0,0){\strut{}DIFFERENT PARAMETERS}}%
    }%
    \gplgaddtomacro\gplfronttext{%
      \csname LTb\endcsname%
      \put(3656,1537){\makebox(0,0)[r]{\strut{}Delta Bits}}%
      \csname LTb\endcsname%
      \put(3656,1317){\makebox(0,0)[r]{\strut{}TABLE\_SIZE}}%
      \csname LTb\endcsname%
      \put(3656,1097){\makebox(0,0)[r]{\strut{}NUM\_DELTAS}}%
      \csname LTb\endcsname%
      \put(3656,877){\makebox(0,0)[r]{\strut{}MAX\_DEGREE}}%
    }%
    \gplbacktext
    \put(0,0){\includegraphics{DCPT-plot}}%
    \gplfronttext
  \end{picture}%
\endgroup

    \caption{Speedup using the DCPT prefetcher}
    \label{graph:dcpt}
\end{center}
\end{figure}
\end{comment}