\section{Methodology}

Hardware components must usually be designed within an area budget, and in order
to simulate realistic conditions, the implemented prefetchers are limited to a
maximum of 8KiB memory~\cite{guidelines}. The simulations are performed by the
M5 simulator available on the Kongull HPC cluster at NTNU.

Both prefetchers are implemented as described, and then simulated with different
configurations in order to seach for patterns and then optimize each prefetcher
to the benchmarks used.

\subsection{The M5 Simulator}

The M5 simulator used in our testing only utilizes a subset of its rich feature
set~\cite{user_doc}, due to the time limitation of this report. The simulator
runs several of the SPEC CPU2000 benchmarks available on the pfJudge course
website~\cite{guidelines}. The benchmarks considered in this report are;
``\emph{ammp}'', ``\emph{applu}'', ``\emph{apsi}'', ``\emph{art110}'',
``\emph{art470}'', ``\emph{bzip2\_graphic}'', ``\emph{bzip2\_program}'',
``\emph{bzip2\_source}'', ``\emph{galgel}'', ``\emph{swim}'', ``\emph{twolf}'',
and ``\emph{wupwise}''.

It is important that the prefetcher performance is evaluated with a variety of
benchmarks representing typical applications. The mentioned benchmarks are
widely used and recognized. We also believe that the results give a proper
foundation for performance evaluation\footnote{For more information about the
benchmarks, please see \emph{www.spec.org}.}.

The prefetcher simulation scores reported in~\ref{sec:res} are averages of the
the speed-up with the abovementioned benchmarks. The speed-up is a measure on
how much faster M5 runs (number of instructions per clock cycle) with one of our
prefetchers compared to no prefetching at all.

We are not extending the simulator any further than what is already done in the
framework described in \ref{section:scheme}.

\subsection{TS Tests}

When testing the TS prefetcher, we vary both the \emph{distance} and the
\emph{degree} parameters it uses. The distances tested are in the range of 2 to
20, and the degree from 1 to 4. The main intention of testing the TS prefetcher
is to have something to compare
the DCPT prefetcher to.

\subsection{DCPT Tests}

When testing the DCPT prefetcher we vary the \emph{table size}, \emph{delta
size} and \emph{delta ring buffer size}.

Initially, these parameters are set to the optimal values found by M. Jahre et
al.~\cite{dcpt}; table size 98, delta size 19 and ring buffer size 12. Although
these values are obtained with different benchmarks and with a maximum 4KiB
prefetcher size, it is a reasonable starting point. The impact of each parameter
is then studied by keeping two of them constant while varying the third.

Since our implementation has 8KiB available, double of M. Jahre et
al.~\cite{dcpt} had, we can exploit this by having larger tables, delta-sizes,
and ring buffer size. Our analysis will reveal wich parameter combinations that
works best.

After this analysis we simulate with the optimal values of each parameter, to
see if it helps the preformance.

\todo[inline]{Explain your experimental setup\\
- How have you extended the simulator?\\
- Which parameters did you use for your simulations? (aim: reproducibility) (DCPT TESTENE ER IKKE REPRODUCIBLE!)\\}
