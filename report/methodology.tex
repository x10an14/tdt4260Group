\section{Methodology}

The size of the prefetcher has been a limitation in our development. Hardware
components must usually be designed within an area budget, and in order to
simulate realistic conditions, the implementation is limited to a maximum of
8KiB memory~\cite{guidelines}. The simulation is performed by the M5 simulator
available on the Kongull HPC cluster at NTNU.

The M5 simulator used in our testing was only utilized with a subset of its rich
feature set~\cite{user_doc}, due to the time limitation of this report.

\todo[inline]{This sentence should belong in where we explain our
implementation. Like the scheme...``The prefetcher utilizes the interface given
in the \emph{interface.hh} file.''}

Table~\ref{tab:test-params} lists the parameters we use in our testing. We chose
these three parameters for comparing our two prefetcher implementations.

\begin{table}[h]
	\caption{Testing parameter variables}
	\begin{tabularx}{\linewidth}{|X|c|X|}
	\hline
	Variable & Value & Description \\
	\hline
	BLOCK\_SIZE & 64 & Size of cache blocks (cache lines) in bytes \\
	\hline
	MAX\_QUEUE\_SIZE & 100 & Maximum number of pending prefetch requests \\
	\hline
	MAX\_PHYS\_MEM\_SIZE & $2^{28}$-1 & The largest possible physical memory address \\
	\hline
	\end{tabularx}
	\label{tab:test-params}
\end{table}

\begin{itemize}
	\item \textbf{BLOCK\_SIZE:} \\
		Since the cache-memory only transfers blocks of memory, we can with this
		variable define the size of the blocks fetched from memory. The larger
		the blocks, the more words, hence less misses due to more words loaded.
		However, if the block sizes are too big, then there cannot be a lot of
		them in the cache at any given time, since it's effectiveness then
		requires the program to perform the majority of its loads/stores from
		the same memory block. So for it to be effective, it requires the
		program not to fetch from different blocks. (Which is unrealistic). The
		optimum is therefore a balanced block size.
	\item \textbf{MAX\_QUEUE\_SIZE:} \\
		Since (to simulate realism) the prefetchers are required to operate
		within the 8KiB memory limit, we cannot hold an infinite amount of
		memory. This includes the list used to decide which memory blocks to
		fetch next. If this list is too large, it may fetch blocks which are not
		yet needed into a full cache, resulting in an eviction of a memory block
		still needed from the cache. If this list is too small, it could result
		in the memory blocks required next not having been fetched into the
		cache at the time needed. Hence, the optimum should be a balanced list
		size, not fetching so far ahead that the cache needs to evict memory
		blocks \textit{still in use}, but neither so that memory blocks
		\textit{needed next} are not already in the cache.
	\item \textbf{MAX\_PHYS\_MEM\_SIZE:}
		Since the prefetcher deals with addresses, the maxiumum size of physical
		memory decides both the addresspace (which has direct consequences for
		the space the prefetcher needs to store these addresses with regards to
		the 8KiB limitation), and how many words (which are then grouped in
		blocks decided by the \textbf{BLOCK\_SIZE} variable) the memory holds.
		The larger this variable is, there is a bigger potential for more
		fetching from different disparate blocks, than if this variable is
		small, results in the opposite; a smaller total of words in the physical
		memory.
\end{itemize}

The M5 simulator runs several of the SPEC CPU2000 benchmarks available on the
pfJudge course website~\cite{guidelines}, so that the prefetcher performance can
be evaluated with different types benchmarks representing typical examples of
different types of applications. The benchmarks considered in this report are;
``\emph{ammp}'', ``\emph{applu}'', ``\emph{apsi}'', ``\emph{art110}'',
``\emph{art470}'', ``\emph{bzip2\_graphic}'', ``\emph{bzip2\_program}'',
``\emph{bzip2\_source}'', ``\emph{galgel}'', ``\emph{swim}'', ``\emph{twolf}'',
``\emph{wupwise}''.

\todo[inline]{Andreas/Ingebrigt: Bruker vi alle disse ovenfornevnte
benchmarkene???\\
Trenger også at noen forteller meg/eller skriver selv, hvorfor
vi bruker de benchmarkene vi ender opp med å bruke!!!}

The score reported in the \textit{Results} section are scores given to our
implemented prefetchers when running the abovementioned benchmarks.

\todo[inline]{Explain your experimental setup\\
- Which simulator did you use?\\
- How have you extended the simulator?\\
- Which parameters did you use for your simulations? (aim: reproducibility)\\
- Which benchmarks did you use?\\
- Why did you chose these benchmarks?\\
$=>$ Important: should be realistic}
