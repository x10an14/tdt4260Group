\section{Methodology}

\subsection{M5 Simulator}

The M5 simulator is a popular simulator for computer architecture research, and
it has a rich feature set \cite{user_doc}. With the given time limitations, we only utilized a subset of it’s complex software during our development.


The prefetcher utilizes an interface given in the \emph{interface.hh} file. 
\begin{table}[h]
	\begin{tabularx}{\linewidth}{|X|c|X|}
	\hline 
	Variable & Value & Description \\ 
	\hline 
	BLOCK\_SIZE & 64 & Size of cache blocks (cache lines) in bytes \\ 
	\hline 
	MAX\_QUEUE\_SIZE & 100 & Maximum number of pending prefetch requests \\ 
	\hline 
	MAX\_PHYS\_MEM\_SIZE & $2^{28}$-1 & The largest possible physical memory address \\ 
	\hline 
	\end{tabularx} 
\end{table}


\todo[inline]{Describe the interface and ``default simulator setup'' better.
Explicitly state the conditions of the default simulator setup/the
functions/structures in the interface.hh file. Don't simply say ``can be studied
there.''\\
\emph{Nico was very clear on this point...}}



\todo[inline]{antar at det er det vi kommer til å gjøre}

The simulator runs several of the SPEC CPU2000 benchmarks, so that the prefetcher
performance can be evaluated with different applications.
The benchmarks considered in this report are ammp, applu, apsi, art110, art470,
bzip2\_graphic, bzip2\_program, bzip2\_source, galgel swim, twolf and wupwise.
The score that is referred to during this report is the average speedup on these benchmarks.


\todo[inline]{Explain your experimental setup\\
- Which simulator did you use?\\
- How have you extended the simulator?\\
- Which parameters did you use for your simulations? (aim: reproducibility)\\
- Which benchmarks did you use?\\
- Why did you chose these benchmarks?\\
$=>$ Important: should be realistic}
