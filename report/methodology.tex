\section{Methodology}

Hardware components must usually be designed within an area budget, and in order
to simulate realistic conditions, the implemented prefetchers are limited to a
maximum of 8KiB memory~\cite{guidelines}. The simulation is performed by the M5
simulator available on the Kongull HPC cluster at NTNU.

The M5 simulator used in our testing was only utilized with a subset of its rich
feature set~\cite{user_doc}, due to the time limitation of this report.

The test result scores reported in the \textit{Results} section are scores given
to our implemented prefetchers when running the abovementioned benchmarks.

When testing the Tagged Sequential prefetcher, we vary both the \emph{distance}
and the \emph{degree} parameters it uses. When testing the DCPT prefetcher we
vary the \emph{table size}, \emph{delta size}, \emph{delta ring buffer size},
and \emph{maximum degree} parameters it uses. For both prefetchers, several
simulations with different combinations of these parameters were run to check
for patterns.

\subsection{The M5 Simulator}

The M5 simulator runs several of the SPEC CPU2000 benchmarks available on the
pfJudge course website~\cite{guidelines}. The benchmarks considered in this
report are; ``\emph{ammp}'', ``\emph{applu}'', ``\emph{apsi}'',
``\emph{art110}'', ``\emph{art470}'', ``\emph{bzip2\_graphic}'',
``\emph{bzip2\_program}'', ``\emph{bzip2\_source}'', ``\emph{galgel}'',
``\emph{swim}'', ``\emph{twolf}'' and ``\emph{wupwise}''.

It is important that the prefetcher performance is evaluated with a variety of
benchmarks representing typical applications. The mentioned benchmarks are
widely used and recognized. We also believe that the results give a proper
foundation for performance evaluation\footnote{For more information about the
benchmarks, please see \emph{www.spec.org}.}.

\todo[inline]{Needs more details? I don't know -C}

\subsection{Tagged Sequential Tests}

\subsection{DCPT Tests}


\todo[inline]{Explain your experimental setup\\
- Which simulator did you use?\\
- How have you extended the simulator?\\
- Which parameters did you use for your simulations? (aim: reproducibility)\\
- Which benchmarks did you use?\\
- Why did you chose these benchmarks?\\
$=>$ Important: should be realistic}
