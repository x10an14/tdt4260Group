\section{Prefetcher Description}

\todo[inline]{The prefetcher description goes here. Another section title maybe?
\\
- Explain your scheme in detail\\
- Choose an informative title\\
- Trick: Add an informative figure that helps explain
your scheme\\
- If your scheme is complex, an informative example
may be in order\\}

\subsection{Delta Correlating Prediction Table}

\subsection{Tagged Sequential}

The tagged sequential prefetcher is a slight improvement over the sequential
prefetcher. Preceeding (etterfølgende?) blocks are prefetched to the cache when
there has been a cache miss, or when a previously prefetched cache entry has
been accessed. \todo{Vet ikke om dette med distance stemmer helt..) Our tagged
sequential prefetcher was implemented with (..........)}The prefetching degree
determines how many blocks should be prefetched at a time, and the distance
specifies how many blocks away from the currently accessed block the prefetches
should be done.

\subsubsection{Hardware requirements}

With the given implementation, a tagged sequential prefetcher simply requires
one bit per cache word in order to store the associated tag. With an L2 cache
size of 1 MB and 64 bit word size, there are 128 different words in the cache,
meaning 128 bits of storage is required for the tagged prefetcher.
