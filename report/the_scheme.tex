\section{The Scheme}

\todo[inline]{NEW STRUCTURE FOR THE SCHEME!!!! AS FOLLOWS:}

\subsection{Simulator Setup}

\todo[inline]{Describing the simulator setup and config values. \\
Describe the ``default simulator setup'' better.
Explicitly state the conditions of the default simulator setup/the
functions/structures in the interface.hh file. Don't simply say ``can be studied
there.''}

\subsection{Tagged Sequential Implementation}

\todo[inline]{Describing our Tagged Sequential implementation. \\
REMEMBER TO EXPLAIN THE INTERFACE!!}

The Tagged Sequential prefetcher can be configured by adjusting two parameters.
The \emph{distance} is how many blocks ahead of the currently accessed block the
prefetcher will fetch, and the \emph{degree} is how many consequtive blocks it will
fetch starting at this location in memory.

With the given implementation, a Tagged Sequential prefetcher simply requires
one bit per cache word in order to store the associated tag. With an L2 cache
size of 1 MB and 64 bit word size, there are 128 different words in the cache,
meaning 128 bits of storage is required for the tagged prefetcher.

The amount of storage used by the DCPT prefetcher can be configured
independently of the cache and memory sizes. The only limitation being that each
entry contains three address fields which must be wide enough to cover the
address space of the system. Apart from that, we can arbitrarily choose the size
and number of deltas, as well as the number of entries in the table.

\todo[inline]{Calculate \% overhead, give the reader an idea of how much 1 MB,
128 bits are in comparison with the limitations.}

\subsection{DCPT Implementation}

\todo[inline]{Describing our DCPT implementation. \\
REMEMBER TO EXPLAIN THE INTERFACE!!}

The DCPT implementation attempts to follow the pseudocode described in the
original implementation~\cite{dcpt}, as closely as possible. However because we
are simulating a hardware implementation, there are some additional checks in
place to deal
\todo[inline]{Nico told us to ignore all things done by compiler. Hence, the below sentence was to be removed he felt. But I didn't write this, so I don't know if this should be removed completely or replaced with something else. -C}
with the fact that the bit sizes some of the fields in the data
structures are larger than they would be in hardware, so we have to limit them
artificially.

The prefetcher can be configured by adjusting four parameters; the number of entries
in the table, the number of deltas per entry and the size in bits of each delta.

\todo[inline]{The following paragraph belongs in the background -ANDREAS-}
The DCPT utilizes a table of ring buffers for keeping track of the deltas, and
so organizing the memory effectively is an important aspect. In the original
implementation~\cite{dcpt} we based our implementation on, only 4KiB of storage
are used.

\todo[inline]{Our prefetcher implementation goes here. Rename the title to ``Our implementation''?\\
- Explain your scheme in detail\\
- Choose an informative title\\
- Trick: Add an informative figure that helps explain
your scheme\\
- If your scheme is complex, an informative example
may be in order\\}
