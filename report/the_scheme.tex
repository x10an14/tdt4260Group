\section{The DCPT And TS Implementation}

\label{section:scheme}

For testing the prefetchers using the M5 simulator, we have been given a simple
framework which takes care of most of the work of setting up and interfacing with
M5. This framework requires us to implement three callbacks. 

\begin{enumerate}
	\item \textbf{prefetch\_init}
		This function is called before the first memory access to let the
		prefetchers initialize their data structures.
	\item \textbf{prefetch\_access}
		This function is called whenever the cache is accessed, both for hits
		and for misses.
	\item \textbf{prefetch\_complete}
		This function is called when a prefetch request has been completed.
\end{enumerate}

\subsection{Simulator Setup}

\todo[inline]{Describing the simulator setup and config values. \\
Describe the ``default simulator setup'' better. Explicitly state the conditions
of the default simulator.}

The simulator is configured with a block size of 64 bytes. The largest possible
memory address is $2^{28}-1$, and up to 100 prefetching requests can be pending
at once.

\subsection{TS Implementation}

For the TS prefetcher is implemented within the functions
\textbf{prefetch\_access} and \textbf{prefetch\_complete}. The 
implementation directly follows the algorithm. 
\textbf{prefetch\_complete} simply sets the prefetch bit on all the 
blocks that are prefetched into the cache, to keep track on the cache content 
that originates from prefetching. When data is accessed by the CPU 
\textbf{prefetch\_access} will check if the 
prefetch bit of the respective data is set. If the bit is set, and if successive 
data is not already in the cache, prefetching will be requested. 
The function also requests prefetching of successive data when a cache 
miss is reported. Prefetching is performed with the given degree 
and distance in both cases.


\subsection{DCPT Implementation}

\todo[inline]{Describing our DCPT implementation}

\todo[inline]{Kan være en godide å skrive om structene vi har selv lagd, hvordan
vi bruker dem, og hva hvilke funksjoner gjør med de (grov oversikt).}

The DCPT implementation attempts to follow the pseudocode described in the
original implementation~\cite{dcpt} as closely as possible.

The prefetcher allocates the data structures statically, so there is nothing to do in \textbf{prefetch\_init}.

The main algorithm is implemented in
\textbf{prefetch\_access}. When the cache is accessed, the prefetcher looks up the entry corresponding to the current program counter (PC), re-initializing the entry if it was currently in use by another instruction. It then compares the current address with the last address accessed by the current instruction and inserts the delta between these addresses into the ring buffer.

The prefetcher then searches the ring buffer for patterns matching the last two deltas, and uses the longest matching sequence of deltas to predict which addresses will be accessed next. It then performs some sanity checks to verify that the candidate addresses are not in the cache, not in the MSHR queue and not already queued for prefetching, before issuing the prefetch of the block containing this address.

In \textbf{prefetch\_complete} the prefetcher removes blocks from the \emph{inFlight} set, used to keep track of blocks which are currently queued for prefetching.

The prefetcher can be configured by adjusting three parameters; the number of
entries in the table, the number of deltas per entry and the size in bits of
each delta.

\subsection{Overhead}

The TS prefetcher requires one bit per cache block to store the
associated tag. Since each cache block is 64 bytes, this amounts to about 0.2\%
overhead.

The amount of storage used by the DCPT prefetcher, however, can be configured
independently of the cache and memory sizes. The only limitation being that each
entry contains three address fields which must be wide enough to cover the
address space of the system. Apart from that, we can arbitrarily choose the size
and number of deltas, as well as the number of entries in the table.

The total storage in bits required by the DCPT implementation is

\begin{equation*}
V = N_e (3 w_a + N_d w_d) + N_f w_a
\end{equation*} 

where $N_e$ is the number of entries in the table, $N_d$ is the number of deltas
per entry, $w_a$ is the number of bits needed to store an address, $w_d$ is the
number of bits per delta, and $N_f$ is the number of entries in the \emph{inFlight}
set.

\todo[inline]{Explain your scheme in detail\\
- Trick: Add an informative figure that helps explain
your scheme\\
- If your scheme is complex, an informative example
may be in order\\}
