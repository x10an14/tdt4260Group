\section{The Scheme}

For testing the prefetchers using the M5 simulator, we implement three callbacks
provided by the framework.

\begin{itemize}
	\item \textbf{prefetch\_init}
		This function is called before the first memory access to let the
        prefetchers initialize their data structures.
	\item \textbf{prefetch\_access}
		This function is called whenever the cache is accessed, both for hits and
        for misses.
	\item \textbf{prefetch\_complete}
        This function is called when a prefetch request has been completed.
\end{itemize}

\subsection{Simulator Setup}

\todo[inline]{Describing the simulator setup and config values. \\
Describe the ``default simulator setup'' better.
Explicitly state the conditions of the default simulator. Don't simply say ``can
be studied there.''}

The simulator is configured with a block size of 64 bytes. The largest possible
memory address is $2^{28}-1$, and up to 100 prefetching requests can be pending at
once.

\subsection{Tagged Sequential Implementation}

\todo[inline]{Describing our Tagged Sequential implementation. \\
REMEMBER TO EXPLAIN THE INTERFACE!!}

For the tagged sequential implementation, we only implement the
\textbf{prefetch\_access} function. The implementation directly follows the
algorithm.

\subsection{DCPT Implementation}

\todo[inline]{Describing our DCPT implementation. \\
REMEMBER TO EXPLAIN THE INTERFACE!!}

The DCPT implementation attempts to follow the pseudocode described in the
original implementation~\cite{dcpt} as closely as possible.

The data structures are allocated statically, so there is nothing to do in
\textbf{prefetch\_init}. The main algorithm is implemented in \textbf{prefetch\_access},
and in \textbf{prefetch\_complete}, we remove blocks from the \emph{inFlight}
set.

The prefetcher can be configured by adjusting four parameters; the number of entries
in the table, the number of deltas per entry and the size in bits of each delta.

\todo[inline]{The following paragraph belongs in the background -ANDREAS-}
The DCPT utilizes a table of ring buffers for keeping track of the deltas, and
so organizing the memory effectively is an important aspect. In the original
implementation~\cite{dcpt} we based our implementation on, only 4KiB of storage
are used.

\todo[inline]{Our prefetcher implementation goes here. Rename the title to ``Our implementation''?\\
- Explain your scheme in detail\\
- Choose an informative title\\
- Trick: Add an informative figure that helps explain
your scheme\\
- If your scheme is complex, an informative example
may be in order\\}

\subsection{Overhead}

The Tagged Sequential prefetcher requires one bit per cache block to store the
associated tag. Since each cache line is 64 bytes, this amounts to about 0.2\%
overhead.

The amount of storage used by the DCPT prefetcher, however, can be configured
independently of the cache and memory sizes. The only limitation being that each
entry contains three address fields which must be wide enough to cover the
address space of the system. Apart from that, we can arbitrarily choose the size
and number of deltas, as well as the number of entries in the table.
