\section{The Scheme}


\todo[inline]{The following paragraph belongs in methodology}
The size of the prefetcher has been a limitation in our development. Hardware
components must usually be designed within an area budget, and in order to
simulate realistic conditions, the implementation is limited to a maximum of
8KiB memory~\cite{guidelines}.

\todo[inline]{Må redgjøre for at vi ikke overskrider denne et sted i rapporten?}

\todo[inline]{The following paragraph belongs in the background}
The DCPT utilizes a table of ring buffers for keeping track of the deltas, and
so organizing the memory effectively is an important aspect. In the original
implementation, only 4KiB of storage are used~\cite{dcpt}.

\todo[inline]{Our prefetcher implementation goes here. Rename the title to ``Our implementation''?\\
- Explain your scheme in detail\\
- Choose an informative title\\
- Trick: Add an informative figure that helps explain
your scheme\\
- If your scheme is complex, an informative example
may be in order\\}

\subsection{Delta Correlating Prediction Table}

The DCPT implementation attempts to follow the pseudocode as closely as possible,
however because we are simulating a hardware implementation, there are some additional
checks in place to deal with the fact that the bit sizes some of the fields in the
data structures are larger than they would be in hardware, so we have to limit them
artificially.

The prefetcher can be configured by adjusting four parameters; the number of entries
in the table, the number of deltas per entry and the size in bits of each delta.

\subsection{Tagged Sequential}

The tagged sequential prefetcher can be configured by adjusting two parameters.
The \emph{distance} is how many blocks ahead of the currently accessed block the
prefetcher will fetch, and the \emph{degree} is how many consequtive blocks it will
fetch starting at this location in memory.

\todo[inline]{The following paragraph belongs in Results}
Our prefetcher was simulated with different combinations of degree and distance,
in the range of 0--6. The combination that got the best overall score on the
benchmark tests was with degree 5 and distance 4.

\subsubsection{Hardware requirements}

With the given implementation, a tagged sequential prefetcher simply requires
one bit per cache word in order to store the associated tag. With an L2 cache
size of 1 MB and 64 bit word size, there are 128 different words in the cache,
meaning 128 bits of storage is required for the tagged prefetcher.

The amount of storage used by the DCPT prefetcher can be configured almost
independently of the cache and memory sizes. The only limitation being that each
entry contains three address fields which must be wide enough to cover the address
space of the system. Apart from that, we can arbitrarily choose the size and
number of deltas, as well as the number of entries in the table. 

\todo[inline]{Calculate \% overhead, give the reader an idea of how much 1 MB,
128 bits are in comparison with the limitations.}
