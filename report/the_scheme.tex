\section{The Scheme}

In this section, we explain our implementation and its setup.

The prefetchers we are creating for this project will be tested by the
previously mentioned M5 simulator. The M5 simulator runs several of the SPEC
CPU2000 benchmarks available on the pfJudge course website~\cite{guidelines}, so
that the prefetcher performance can be evaluated with different types benchmarks
representing typical examples of different types of applications. Only a subset
of M5's rich feature set~\cite[Ch.~2]{user_doc} was utilized in our testing,
due to the time limitation of this report.

Both the Tagged Sequential prefetcher and the DCPT prefetcher were implemented
based on the following three missing functions in the framework provided to us
by the course\cite[Ch.~3]{user_doc}.
\begin{enumerate}
	\item \textbf{void prefetch\_init(void)}
		This function will be called before any memory access, to let our
		implemeneted prefetchers initialize their data structures.
	\item \textbf{void prefetch\_access(AccessStat stat)}
		This function tells the implemented prefetcher about a cache access.
	\item \textbf{void prefetch\_complete(Addr addr)}
		This function notifies the implemented prefetcher about a prefetch load
		that has just completed.
\end{enumerate}

In the framework our prefetchers are implemented in, a cache block is 64 Bytes
large. Our prefetcher is used for the Level 2 (L2) cache, which is 1MiBs large.
This means that the L2 cache has space for a total of $16$~$384$ cache blocks.

\subsection{Simulator Setup}

\todo[inline]{Describing the simulator setup and config values. \\
Describe the ``default simulator setup'' better.
Explicitly state the conditions of the default simulator. Don't simply say ``can
be studied there.''}

Table~\ref{tab:prefetch-consts} lists the limits of the parameters the m5
simulator has.
\begin{table}[h]
	\caption{Interface definitions}
	\begin{tabularx}{\linewidth}{|X|c|X|}
	\hline
	Variable & Value & Description \\
	\hline
	BLOCK\_SIZE & 64 & Size of cache blocks (cache lines) in bytes \\
	\hline
	MAX\_QUEUE\_SIZE & 100 & Maximum number of pending prefetch requests \\
	\hline
	MAX\_PHYS\_MEM\_SIZE & $2^{28}$-1 & The largest possible physical memory address \\
	\hline
	\end{tabularx}
	\label{tab:prefetch-consts}
\end{table}

\begin{itemize}
	\item \textbf{BLOCK\_SIZE:} \\
		Since the cache-memory only transfers blocks of memory, we can with this
		variable define the size of the blocks fetched from memory. The larger
		the blocks, the more words, hence less misses due to more words loaded.
		However, if the block sizes are too big, then there cannot be a lot of
		them in the cache at any given time, since it's effectiveness then
		requires the program to perform the majority of its loads/stores from
		the same memory block. So for it to be effective, it requires the
		program not to fetch from different blocks. (Which is unrealistic). The
		optimum is therefore a balanced block size.
	\item \textbf{MAX\_QUEUE\_SIZE:} \\
		Since (to simulate realism) the prefetchers are required to operate
		within the 8KiB memory limit, we cannot hold an infinite amount of
		memory. This includes the list used to decide which memory blocks to
		fetch next. If this list is too large, it may fetch blocks which are not
		yet needed into a full cache, resulting in an eviction of a memory block
		still needed from the cache. If this list is too small, it could result
		in the memory blocks required next not having been fetched into the
		cache at the time needed. Hence, the optimum should be a balanced list
		size, not fetching so far ahead that the cache needs to evict memory
		blocks \textit{still in use}, but neither so that memory blocks
		\textit{needed next} are not already in the cache.
	\item \textbf{MAX\_PHYS\_MEM\_SIZE:}
		Since the prefetcher deals with addresses, the maxiumum size of physical
		memory decides both the addresspace (which has direct consequences for
		the space the prefetcher needs to store these addresses with regards to
		the 8KiB limitation), and how many words (which are then grouped in
		blocks decided by the \textbf{BLOCK\_SIZE} variable) the memory holds.
		The larger this variable is, there is a bigger potential for more
		fetching from different disparate blocks, than if this variable is
		small, results in the opposite; a smaller total of words in the physical
		memory.
\end{itemize}

\subsection{Tagged Sequential Implementation}

\todo[inline]{Describing our Tagged Sequential implementation. \\
REMEMBER TO EXPLAIN THE INTERFACE!!}

With the given implementation, a Tagged Sequential prefetcher simply requires
one bit per cacheword to store the associated tag. With our L2 cache size, cache block size, and word size, this equates to needing $131$~$072$ tag bits to
represent all the cache words. This amount equates to 16KiB of memory, or in
other words $\frac{1}{512} \approx 0.2\%$ of the L2 cache used for overhead.

The amount of storage used by the DCPT prefetcher however can be configured
independently of the cache and memory sizes. The only limitation being that each
entry contains three address fields which must be wide enough to cover the
address space of the system. Apart from that, we can arbitrarily choose the size
and number of deltas, as well as the number of entries in the table.

\subsection{DCPT Implementation}

\todo[inline]{Describing our DCPT implementation. \\
REMEMBER TO EXPLAIN THE INTERFACE!!}

The DCPT implementation attempts to follow the pseudocode described in the
original implementation~\cite{dcpt}, as closely as possible. However because we
are simulating a hardware implementation, there are some additional checks in
place to deal
\todo[inline]{Nico told us to ignore all things done by compiler. Hence, the below sentence was to be removed he felt. But I didn't write this, so I don't know if this should be removed completely or replaced with something else. -C}
with the fact that the bit sizes some of the fields in the data
structures are larger than they would be in hardware, so we have to limit them
artificially.

The prefetcher can be configured by adjusting four parameters; the number of entries
in the table, the number of deltas per entry and the size in bits of each delta.

\todo[inline]{The following paragraph belongs in the background -ANDREAS-}
The DCPT utilizes a table of ring buffers for keeping track of the deltas, and
so organizing the memory effectively is an important aspect. In the original
implementation~\cite{dcpt} we based our implementation on, only 4KiB of storage
are used.

\todo[inline]{Our prefetcher implementation goes here. Rename the title to ``Our implementation''?\\
- Explain your scheme in detail\\
- Choose an informative title\\
- Trick: Add an informative figure that helps explain
your scheme\\
- If your scheme is complex, an informative example
may be in order\\}
